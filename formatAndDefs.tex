\usepackage{amsmath,amssymb}             % AMS Math
% \usepackage[french]{babel}
\usepackage[latin1]{inputenc}
\usepackage[T1]{fontenc}
\usepackage[left=0.9in,right=0.8in,top=0.8in,bottom=0.8in,includefoot,includehead,headheight=13.6pt]{geometry}
\renewcommand{\baselinestretch}{1.05}

%\renewcommand{\thechapter}{\Roman{chapter}}

% Table of contents for each chapter

\usepackage[nottoc, notlof, notlot]{tocbibind}
\usepackage{minitoc}
\setcounter{minitocdepth}{2}
\mtcindent=15pt
% Use \minitoc where to put a table of contents

\usepackage{aecompl}

% Glossary / list of abbreviations

\usepackage[intoc]{nomencl}
\renewcommand{\nomname}{List of Abbreviations}

\makenomenclature

% My pdf code

\usepackage{ifpdf}

\ifpdf
  \usepackage[pdftex]{graphicx}
  \DeclareGraphicsExtensions{.jpg}
  \usepackage[a4paper,pagebackref,hyperindex=true]{hyperref}
\else
  \usepackage{graphicx}
  \DeclareGraphicsExtensions{.ps,.eps}
  \usepackage[a4paper,dvipdfm,pagebackref,hyperindex=true]{hyperref}
\fi

\graphicspath{{.}{images/}}

% nicer backref links
\renewcommand*{\backref}[1]{}
\renewcommand*{\backrefalt}[4]{%
\ifcase #1 %
(Not cited.)%
\or
(Cited on page~#2.)%
\else
(Cited on pages~#2.)%
\fi}
\renewcommand*{\backrefsep}{, }
\renewcommand*{\backreftwosep}{ and~}
\renewcommand*{\backreflastsep}{ and~}

% Links in pdf
\usepackage{color}
\definecolor{linkcol}{rgb}{0,0,0.4} 
\definecolor{citecol}{rgb}{0.5,0,0} 
\definecolor{titlepage}{rgb}{0.5,0,0}

% Change this to change the informations included in the pdf file

% See hyperref documentation for information on those parameters

\hypersetup
{
bookmarksopen=true,
pdftitle="To define",
pdfauthor="J�r�my DUBUT", 
pdfsubject="Directed algebraic topology", %subject of the document
%pdftoolbar=false, % toolbar hidden
pdfmenubar=true, %menubar shown
pdfhighlight=/O, %effect of clicking on a link
colorlinks=true, %couleurs sur les liens hypertextes
pdfpagemode=None, %aucun mode de page
pdfpagelayout=SinglePage, %ouverture en simple page
pdffitwindow=true, %pages ouvertes entierement dans toute la fenetre
linkcolor=linkcol, %couleur des liens hypertextes internes
citecolor=citecol, %couleur des liens pour les citations
urlcolor=linkcol %couleur des liens pour les url
}

% definitions.
% -------------------

\setcounter{secnumdepth}{3}
\setcounter{tocdepth}{2}

% Some useful commands and shortcut for maths:  partial derivative and stuff

\newcommand{\pd}[2]{\frac{\partial #1}{\partial #2}}
\def\abs{\operatorname{abs}}
\def\argmax{\operatornamewithlimits{arg\,max}}
\def\argmin{\operatornamewithlimits{arg\,min}}
\def\diag{\operatorname{Diag}}
\newcommand{\eqRef}[1]{(\ref{#1})}
%\newcommand\bf[1]{\textbf{#1}}
\newcommand\map[3]{#1 : #2 \longrightarrow #3}
\newcommand\pathto[3]{#1 : #2 \leadsto #3}
\newcommand\tr{\textbf{Tr(}\Sigma\textbf{)}}
\newcommand\astr{\textbf{ATr(}\Sigma\textbf{)}}
\newcommand\dessin{\textcolor{red}{FAIRE UN DESSIN}}
\newcommand\br{\textbf{Br(}\Sigma\textbf{)}}
\newcommand\setcat{\textbf{Set}}
\newcommand\group{\textbf{Gr}}
\newcommand\sset{\textbf{SSet}}
\newcommand\C{\mathcal{C}}
\newcommand\D{\mathcal{D}}
\newcommand\A{\mathcal{A}}
\newcommand\M{\mathcal{M}}
\newcommand\PP{\mathcal{P}}
\newcommand\E{\mathcal{E}}
\newcommand\R{\mathcal{R}}
\newcommand\X{\mathcal{X}}
\newcommand\V{\mathcal{V}}
\newcommand\RR{\mathbb{R}}
\newcommand\QQ{\mathbb{Q}}
\newcommand\ZZ{\mathbb{Z}}
\newcommand\diagr[1]{\textbf{Diag(}#1\textbf{)}}
\newcommand\diagriso[1]{\textbf{IsoDiag(}#1\textbf{)}}
\newcommand\podiagr[1]{\textbf{PoDiag(}#1\textbf{)}}
\newcommand\enrcat[1]{\textbf{Cat(}#1\textbf{)}}
\newcommand\pecat[1]{\textbf{PeCat(}#1\textbf{)}}
\newcommand\nat{\mathbb{N}}
\newcommand\Ab{\textbf{Ab}}
\newcommand\cat{\textbf{Cat}}
\newcommand\catpair{\textbf{Cat}_2}
\newcommand\grpd{\textbf{Grpd}}
\newcommand\Unf[1]{\text{Unf}(#1)}
\newcommand\Unfo{\text{Unf}}
\newcommand\ob[1]{\text{Ob}(#1)}
\newcommand\PN{\textbf{PN(}\Sigma\textbf{)}}
\newcommand\hotop{\textbf{HoTop}}
\newcommand\topo{\textbf{Top}}
\newcommand\geo[1]{\text{Geom}(#1)}
\newcommand\psh[1]{\textbf{PSh}(#1)}
\newcommand\pathsp[1]{\text{P}(#1)}
\newcommand\pathspp[3]{\text{P}(#1)(#2,#3)}
\newcommand\potop{\textbf{POTop}}
\newcommand\lpotop{\textbf{LocPOTop}}
\newcommand\prestr{\textbf{PreStr}}
\newcommand\str{\textbf{Str}}
\newcommand\dirseg{\overrightarrow{[0,1]}}
\newcommand\dirsegrev{\overleftarrow{[0,1]}}
\newcommand\revseg{\widetilde{[0,1]}}
\newcommand\dip[1]{\overrightarrow{\text{P}}(#1)}
\newcommand\rev[1]{\widetilde{\text{P}}(#1)}
\newcommand\dipp[3]{\overrightarrow{\text{P}}(#1)(#2,#3)}
\newcommand\dtop{\textbf{dTop}}
\newcommand\cdtop{\textbf{CdTop}}
\newcommand\hstr{\textbf{HStr}}
\newcommand\matchbox{\mathbb{M}_\Box}
\newcommand\funcat[1]{\overrightarrow{\pi_1}(#1)}
\newcommand\loca[2]{#1[#2^{-1}]}
\newcommand\ine[1]{\mathfrak{I}(#1)}
\newcommand\spa[5]{#1 \xleftarrow{~#2~} #3 \xrightarrow{~#4~} #5}
\newcommand\comcat[1]{\overrightarrow{\pi_0}(#1)}
\newcommand\topp{\textbf{Top}_2}
\newcommand\trace[1]{\overrightarrow{T}(#1)}
\newcommand\ttrace[1]{\overrightarrow{\mathbb{T}}(#1)}
\newcommand\tracep[3]{\overrightarrow{T}(#1)(#2,#3)}
\newcommand\tra[1]{\langle#1\rangle}
\newcommand\pale[3]{#1 : #2 \leadsto #3}
\newcommand\modu[1]{\textbf{Mod}(#1)}
\newcommand\chain[1]{\textbf{C}_\bullet(#1)}
\newcommand\env[1]{\mathcal{E}(#1)}
\newcommand\tracehm[1]{\overrightarrow{BT}(#1)}
\newcommand\fac[1]{\mathcal{F}(#1)}
\newcommand\tracebw[1]{\overrightarrow{NT}(#1)}
\newcommand\nathomot[2]{\overrightarrow{\Pi}_{#1}(#2)}
\newcommand\nathomolbw[2]{\overrightarrow{NH}_{#1}(#2)}
\newcommand\nathomolhm[2]{\overrightarrow{BH}_{#1}(#2)}
\newcommand\dipnathomolhm[2]{\overrightarrow{DBH}_{#1}(#2)}
\newcommand\chainbw[1]{\overrightarrow{NC}(#1)}
\newcommand\chainhm[1]{\overrightarrow{BC}(#1)}
\newcommand\Lim[1]{\text{Lim}(#1)}
\newcommand\id{\text{id}}
\newcommand\ke[1]{\text{ker }#1}
\newcommand\Ke[1]{\text{Ker }#1}
\newcommand\cok[1]{\text{cok }#1}
\newcommand\Cok[1]{\text{Cok }#1}
\newcommand\Ima[1]{\text{Im }#1}
\newcommand\ima[1]{\text{im }#1}
\newcommand\coima[1]{\text{coim }#1}
\newcommand\Coima[1]{\text{Coim }#1}
\newcommand\diagrmod[1]{\diagr{\modu{#1}}}
\newcommand\quot[2]{{#1}/{#2}}
\newcommand\digeom[1]{\overrightarrow{Geom}(#1)}
\newcommand\disnathomolhm[2]{\overrightarrow{DBH}_{#1}(#2)}
\newcommand\disnathomolbw[2]{\overrightarrow{DNH}_{#1}(#2)}
\newcommand\carrier{\text{Car}}
\newcommand\subd[1]{\text{Sub}(#1)}
\newcommand\dicir{\overrightarrow{S^1}}
\newcommand\fib{\text{Fib}}
\newcommand\cof{\text{Cof}}
\newcommand\sing{\text{Sing}}
\newcommand\pspan[5]{#2 \xleftarrow{~~ #4 ~~} #1 \xrightarrow{~~ #5 ~~} #3}
\newcommand\presh[1]{[#1^{op},Set]}
\newcommand\yon[0]{\mathfrak{Y}}
\newcommand\grp[0]{\textbf{Grpd}_\star}
\newcommand\I[0]{\mathcal{I}}
\newcommand\tre[2]{\textbf{Tree(}#1,#2\textbf{)}}



%\input{AckFor}



\definecolor{ultramarineblue}{rgb}{0.39,0.58,0.93}
\definecolor{fieryorange}{RGB}{236,117,40}
\definecolor{blackorkgreen}{rgb}{0.13,0.55,0.13}
\definecolor{badmoonyellow}{rgb}{1,0.84,0}


\usepackage{amsmath}
\usepackage{amssymb}
\usepackage{amsthm}
\usepackage{shuffle}

\renewcommand\qedsymbol{$.QED.$}


\theoremstyle{plain}
\newtheorem{lemme}{Lemma} 
\newtheorem{theo}{Theorem}
\newtheorem{conj}{Conjecture} 
\newtheorem{prop}{Proposition}    
\newtheorem{coro}{Corollary}
\theoremstyle{definition}
\newtheorem{defi}{Definition}
\newtheorem{nota}{Notation}
\newtheorem{exe}{Example}
\newtheorem{exes}{Exemples}
\theoremstyle{remark}
\newtheorem{rem}{Remarque}



\usepackage{tikz}
\usetikzlibrary{decorations.pathmorphing}
\usetikzlibrary{shapes}
\usetikzlibrary{matrix}
\usepackage{xifthen}
\tikzset{snake it/.style={decorate, decoration=snake}}
\usepackage[hang,small]{caption}
\usepackage{pgfplots}
\usepackage{pdfpages}
%\usepackage{algorithm,algorithmic}

\usepackage{rotating}                    % Sideways of figures & tables
%\usepackage{bibunits}
%\usepackage[sectionbib]{chapterbib}          % Cross-reference package (Natural BiB)
%\usepackage{natbib}                  % Put References at the end of each chapter
                                         % Do not put 'sectionbib' option here.
                                         % Sectionbib option in 'natbib' will do.
\usepackage{fancyhdr}                    % Fancy Header and Footer

% \usepackage{txfonts}                     % Public Times New Roman text & math font
  
%%% Fancy Header %%%%%%%%%%%%%%%%%%%%%%%%%%%%%%%%%%%%%%%%%%%%%%%%%%%%%%%%%%%%%%%%%%
% Fancy Header Style Options

\pagestyle{fancy}                       % Sets fancy header and footer
\fancyfoot{}                            % Delete current footer settings

%\renewcommand{\chaptermark}[1]{         % Lower Case Chapter marker style
%  \markboth{\chaptername\ \thechapter.\ #1}}{}} %

%\renewcommand{\sectionmark}[1]{         % Lower case Section marker style
%  \markright{\thesection.\ #1}}         %

\fancyhead[LE,RO]{\bfseries\thepage}    % Page number (boldface) in left on even
% pages and right on odd pages
\fancyhead[RE]{\bfseries\nouppercase{\leftmark}}      % Chapter in the right on even pages
\fancyhead[LO]{\bfseries\nouppercase{\rightmark}}     % Section in the left on odd pages

\let\headruleORIG\headrule
\renewcommand{\headrule}{\color{black} \headruleORIG}
\renewcommand{\headrulewidth}{1.0pt}
\usepackage{colortbl}
\arrayrulecolor{black}

\fancypagestyle{plain}{
  \fancyhead{}
  \fancyfoot{}
  \renewcommand{\headrulewidth}{0pt}
}

\usepackage{algorithm}
%\usepackage[noend]{algorithmic}
\usepackage{algpseudocode}

%%% Clear Header %%%%%%%%%%%%%%%%%%%%%%%%%%%%%%%%%%%%%%%%%%%%%%%%%%%%%%%%%%%%%%%%%%
% Clear Header Style on the Last Empty Odd pages
\makeatletter

\def\cleardoublepage{\clearpage\if@twoside \ifodd\c@page\else%
  \hbox{}%
  \thispagestyle{empty}%              % Empty header styles
  \newpage%
  \if@twocolumn\hbox{}\newpage\fi\fi\fi}

\makeatother
 
%%%%%%%%%%%%%%%%%%%%%%%%%%%%%%%%%%%%%%%%%%%%%%%%%%%%%%%%%%%%%%%%%%%%%%%%%%%%%%% 
% Prints your review date and 'Draft Version' (From Josullvn, CS, CMU)
\newcommand{\reviewtimetoday}[2]{\special{!userdict begin
    /bop-hook{gsave 20 710 translate 45 rotate 0.8 setgray
      /Times-Roman findfont 12 scalefont setfont 0 0   moveto (#1) show
      0 -12 moveto (#2) show grestore}def end}}
% You can turn on or off this option.
% \reviewtimetoday{\today}{Draft Version}
%%%%%%%%%%%%%%%%%%%%%%%%%%%%%%%%%%%%%%%%%%%%%%%%%%%%%%%%%%%%%%%%%%%%%%%%%%%%%%% 

\newenvironment{maxime}[1]
{
\vspace*{0cm}
\hfill
\begin{minipage}{0.5\textwidth}%
%\rule[0.5ex]{\textwidth}{0.1mm}\\%
\hrulefill $\:$ {\bf #1}\\
%\vspace*{-0.25cm}
\it 
}%
{%

\hrulefill
\vspace*{0.5cm}%
\end{minipage}
}

\let\minitocORIG\minitoc
\renewcommand{\minitoc}{\minitocORIG \vspace{1.5em}}

\usepackage{multirow}
\usepackage{slashbox}

\newenvironment{bulletList}%
{ \begin{list}%
	{$\bullet$}%
	{\setlength{\labelwidth}{25pt}%
	 \setlength{\leftmargin}{30pt}%
	 \setlength{\itemsep}{\parsep}}}%
{ \end{list} }

\newtheorem{definition}{D�finition}
\renewcommand{\epsilon}{\varepsilon}

% centered page environment

\newenvironment{vcenterpage}
{\newpage\vspace*{\fill}\thispagestyle{empty}\renewcommand{\headrulewidth}{0pt}}
{\vspace*{\fill}}

